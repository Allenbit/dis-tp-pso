%%%%%%%%%%%%%%%%%%%%%%%%%%%%%%%%%%%%%%%%%%%%%%%%%%%%%%%%%%%%%%%%%%%%%%%%%%%%%%%%
%2345678901234567890123456789012345678901234567890123456789012345678901234567890
%        1         2         3         4         5         6         7         8

\documentclass[a4, 10 pt, conference]{ieeeconf}  % Comment this line out
                                                          % if you need a4paper
%\documentclass[a4paper, 10pt, conference]{ieeeconf}      % Use this line for a4
                                                          % paper

\IEEEoverridecommandlockouts                              % This command is only
                                                          % needed if you want to
                                                          % use the \thanks command
\overrideIEEEmargins
% See the \addtolength command later in the file to balance the column lengths
% on the last page of the document



% The following packages can be found on http:\\www.ctan.org
%\usepackage{graphics} % for pdf, bitmapped graphics files
%\usepackage{epsfig} % for postscript graphics files
%\usepackage{mathptmx} % assumes new font selection scheme installed
%\usepackage{times} % assumes new font selection scheme installed
%\usepackage{amsmath} % assumes amsmath package installed
%\usepackage{amssymb}  % assumes amsmath package installed

\title{\LARGE \bf
Learning a ‘following and chain-building’ behavior using PSO
}

%\author{ \parbox{3 in}{\centering Huibert Kwakernaak*
%         \thanks{*Use the $\backslash$thanks command to put information here}\\
%         Faculty of Electrical Engineering, Mathematics and Computer Science\\
%         University of Twente\\
%         7500 AE Enschede, The Netherlands\\
%         {\tt\small h.kwakernaak@autsubmit.com}}
%         \hspace*{ 0.5 in}
%         \parbox{3 in}{ \centering Pradeep Misra**
%         \thanks{**The footnote marks may be inserted manually}\\
%        Department of Electrical Engineering \\
%         Wright State University\\
%         Dayton, OH 45435, USA\\
%         {\tt\small pmisra@cs.wright.edu}}
%}

\author{Alice Concordel$^{1}$, Christoph K\"{o}rner$^{2}$ and Etienne Thalmann$^{3}$% <-this % stops a space
\thanks{This work is a project of the course Distributed intelligent systems by Prof. Alcherio Martinoli}% <-this % stops a space
\thanks{A. Concordel is with Faculty of Mechanical Engineering, School of engineering, Ecole Polytechnique fédérale de Lausanne}
\thanks{C. K\"{o}rner is with the Faculty of Electrical Engineering, Vienna University of Technology} %=============================FILL IN!!!
\thanks{E. Thalmann is with Faculty of Mechanical Engineering, School of engineering, Ecole Polytechnique fédérale de Lausanne}%
}

\begin{document}

\maketitle
\thispagestyle{empty}
\pagestyle{empty}


%%%%%%%%%%%%%%%%%%%%%%%%%%%%%%%%%%%%%%%%%%%%%%%%%%%%%%%%%%%%%%%%%%%%%%%%%%%%%%%%
\begin{abstract}


\end{abstract}


%%%%%%%%%%%%%%%%%%%%%%%%%%%%%%%%%%%%%%%%%%%%%%%%%%%%%%%%%%%%%%%%%%%%%%%%%%%%%%%%
\section{INTRODUCTION}
The goal of this project is to implement a braitenberg-type controller for movement in
formation on epucks. For this, a PSO algorithm was implemented, using the braitenberg weights as a search space. The formation consists of 3 follower robots and one leader that moves with a predefined trajectory.

\subsection{Design of PSO}
In order to design a PSO algorithm for a collaborative task, we had to consider three axes: The diversity of the team, the performance evaluation and the solution sharing. We decided  to use a heterogeneous approach with individual performance evaluation and group solution sharing. The idea is to evaluate different solutions on the three followers (heterogenous approach) in order to save computation time and to share the pool of solutions (group solution) to have evolution of the particles. Since we are interested in the performance of each follower with its own particle and not in the performance of the entire chain we need an individual performance evaluation.


\subsection{Design of Fitness Function}
The idea of the fitness function is to evaluate the performance of a controller in being able to follow a leader and stay in chain formation. The design of the fitness function is of primary importance because it defines towards which solution the PSO will converge. Four characteristics were found to be important to evaluate following and chain-building behavior:
\begin{itemize}
\item The range $d$, which is the distance between the follower and the leader, has to be minimized.
\item The relative heading of leader and follower $\Delta \phi $. The heading is defined as the angular orientation of the robot in a fixed frame of reference. The relative heading is zero, when both robots are facing the same way. This angle has to be minimized in order to have nice following. 
\item The bearing $\theta $, which is the angular offset of the leader's position with respect to the heading of the follower. This quantity has to be minimized so that the follower always stays behind the leader.
\item The relative speed between leader and follower $\Delta v $ can be minimized in order to have a smoother following behavior. This was part of our original design but has later been proven unnecessary.
\end{itemize}
The fitness function has to be maximized. We define it as follows:
\begin{equation}\label{fitness}
F=\frac{1}{A*d+B*\theta+C*\Delta phi+C* \Delta v}
\end{equation}
Where A,B,C,D are coefficients of importance of the different components that are initially set to one and have to be determined.

\section{Methodology}

%\begin{table}
%\caption{An Example of a Table}
%\label{table_example}
%\begin{center}
%\begin{tabular}{|c||c|}
%\hline
%One & Two\\
%\hline
%Three & Four\\
%\hline
%\end{tabular}
%\end{center}
%\end{table}



\addtolength{\textheight}{-3cm}   % This command serves to balance the column lengths
                                  % on the last page of the document manually. It shortens
                                  % the textheight of the last page by a suitable amount.
                                  % This command does not take effect until the next page
                                  % so it should come on the page before the last. Make
                                  % sure that you do not shorten the textheight too much.
                                  
\subsection{Implementation Webots}

\subsubsection{Environment}

\subsubsection{Leader}
\subsubsection{Supervisor}
The role of the supervisor is to set the initial settings and then to run the PSO. In short, it sends the particles to the followers for their evaluation, gets their fitness back, evolves the swarm and restarts an iteration.\\
The number of particles in the swarm is set higher than the number of followers. Since the approach is heterogeneous, the supervisor evaluates them in batches of three by sending one particle (i.e. one set of weights for their Braitenberg controller) to each follower. Then comes the evaluation period during which the followers evaluate the fitness of the controller. To evaluate the fitness of its run according to equation \ref{fitness}, the follower needs to know its coordinates and those of the robot it is following. Thus, while the followers are doing their evaluation run, the supervisor sends to the followers their spatial coordinates (translation and roatation) and those of their respective leader. Once the fitness of each particles is evaluated, it is returned to the supervisor and stored. The supervisor resets the followers and the leader in their initial position and replicates the previous procedure with the following batch of particles. Once all the particles of the swarm have been evaluated, they evolve according to the PSO algorithm using the fitness values obtained for each particle. The new swarm obtained is evaluated  and evolved in the same way as the previous swarm. The number of these evolution iterations is set beforehand.\\
When the last iteration is over, the weights of the particle that returned the best fitness during this run of the PSO are tested by sending them to the followers a chosen number of times and measuring the average fitness of the solution. If the results are better than the previous best, the solution is stored in a text file and set as initial weights for the next PSO run. The entire PSO algorithm is run again like the previous one, each particle being initialized with the best weights so far. This higher level of iterations is run as long as needed to get satisfying results. The weights that had the best fitness evaluation can be extracted from the text file and implemented on the real epucks.\\
The supervisor also communicates in a unilateral way with the leader, sending an interrupt message every time the positions are reset. The purpose of this is that the leader restarts its trajectory, so as to have always the same trajectory in the case of predefined ones. This was needed at the beginning of the evolutions in order to have all the particles evaluated on a trajectory having the same level of difficulty. This way, weights that are worse than others cannot have a better fitness evaluation just because the trajectory was easier, which would be counter productive.

\subsubsection{Follower}

\subsection{PSO characteristics and Optimization}
swarmsize, neigborhoods, iterations, initial weights etc.
optiomize the fitness etc noise resistant
no penalty fo rmax speed etc
bearing more important... ABCD




   \begin{figure}[thpb]
      \centering
      %\includegraphics[scale=1.0]{figurefile}
      \caption{Inductance of oscillation winding on amorphous
       magnetic core versus DC bias magnetic field}
      \label{figurelabel}
   \end{figure}

%%%%%%%%%%%%%%%%%%%%%%%%%%%%%%%%%%%%%%%%%%%%%%%%%%%%%%%%%%%%%%%%%%%%%%%%%%%%%%%%
\section{Results}
trajectories and iterations
%%%%%%%%%%%%%%%%%%%%%%%%%%%%%%%%%%%%%%%%%%%%%%%%%%%%%%%%%%%%%%%%%
\section{Conclusions}

%%%%%%%%%%%%%%%%%%%%%%%%%%%%%%%%%%%%%%%%%%%%%%%%%%%%%%%%%%%%%%%%%%%%%%%%%%%%%%%%
\section{ACKNOWLEDGMENTS}
The authors would like to thank Alcherio Martinoli for giving the course, Milos Vasic, for his precious advice and support, as well as all the other teaching assistants.

\begin{thebibliography}{99}

\bibitem{c1}
J.G.F. Francis, The QR Transformation I, {\it Comput. J.}, vol. 4, 1961, pp 265-271.

\bibitem{c2}
H. Kwakernaak and R. Sivan, {\it Modern Signals and Systems}, Prentice Hall, Englewood Cliffs, NJ; 1991.

\bibitem{c3}
D. Boley and R. Maier, "A Parallel QR Algorithm for the Non-Symmetric Eigenvalue Algorithm", {\it in Third SIAM Conference on Applied Linear Algebra}, Madison, WI, 1988, pp. A20.

\end{thebibliography}

\end{document}
